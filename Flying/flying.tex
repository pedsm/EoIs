\documentclass[11pt, oneside]{article}
\title{Expression of Interest}
\author{Group 1}
\usepackage{graphicx}
\sloppy
\usepackage[utf8]{inputenc}
\usepackage[parfill]{parskip}
\usepackage{geometry}
\usepackage{multicol}
\usepackage[parfill]{parskip}
\usepackage{graphicx}
\usepackage[rightcaption]{sidecap} \usepackage[hyphens]{url}
\usepackage[hidelinks]{hyperref}
\usepackage{amssymb}
\usepackage{fullpage}
\usepackage{color}
\usepackage[dvipsnames]{xcolor}

%Computer Science Packages for code and algorithms
\usepackage{listings}
\usepackage[]{algorithm2e}

\definecolor{codegreen}{rgb}{0.32,0.75,0.35}
\definecolor{codegray}{rgb}{0.5,0.5,0.5}
\definecolor{codebrown}{rgb}{0.72,0.54,0}
\definecolor{backcolour}{rgb}{0,0.17,0.26}
\newcommand{\lastline}[1]{%
    \begingroup\setlength{\parskip}{0pt}\par\nopagebreak
        \raggedleft#1\par\endgroup
        }

\lstdefinestyle{darkstyle}{
    backgroundcolor=\color{backcolour},   
    commentstyle=\color{codegreen},
    keywordstyle=\color{magenta},
    numberstyle=\ttfamily\tiny\color{backcolour},
    stringstyle=\color{ProcessBlue},
    basicstyle=\ttfamily\footnotesize\color{white},
    breakatwhitespace=false,         
    captionpos=b,                    
    keepspaces=true,                 
    numbers=left,                    
    numbersep=5pt,                  
    showspaces=false,                
    showstringspaces=false,
    showtabs=false,                  
    tabsize=2
}
\lstdefinestyle{whitestyle}{
    commentstyle=\color{Emerald},
    keywordstyle=\color{red},
    numberstyle=\ttfamily\tiny\color{backcolour},
    stringstyle=\color{ProcessBlue},
    basicstyle=\ttfamily\footnotesize\color{black},
    breakatwhitespace=false,         
    captionpos=b,                    
    keepspaces=true,                 
    numbers=left,                    
    numbersep=5pt,                  
    showspaces=false,                
    showstringspaces=false,
    showtabs=false,                  
    tabsize=2
}
\lstset{language=Haskell}
\lstset{breaklines=true}
\lstset{style=whitestyle}


\PassOptionsToPackage{hyphens}{url}\usepackage{hyperref}
\geometry{a4paper}

%header
% \usepackage{fancyhdr}
% \setlength{\headheight}{28pt}
% \pagestyle{fancy}
% \fancyhf{}
% \rhead{Pedro Santos de Mendonca}
% \rfoot{\thepage}
% \renewcommand{\headrulewidth}{1pt}
% \setlength{\topmargin}{10pt}
% \setlength{\headsep}{20pt}

%\usepackage{setspace}
%\doublespacing
\usepackage{mathtools}

\usepackage[english]{babel}

\begin{document}
\maketitle
\textbf{Project Title: }What's flying around me?\\
% \textbf{Organisation or Supervisor}Isaac Triguero\newline
\textbf{Contact person: }Steven Bagley\newline
\textbf{Contact email: }srb@cs.notts.ac.uk\\
    
\vspace{0.3cm}
\textbf{Team Members}\newline
\begin{minipage}[t]{0.5\textwidth}
    \textbf{Name}\\
    Pedro Santos de Mendonca\\
    Jonathan Balls\\
    Samuel Copping\\
    Pengsan Wong\\
    Holly Tang\\
    Huachen Zhang\\
\end{minipage}
\begin{minipage}[t]{0.5\textwidth}
    \textbf{Email}\\
    psyps4@nottingham.ac.uk \\
    psycjba@nottingham.ac.uk \\
    psysrc@nottingham.ac.uk \\
    psy@nottingham.ac.uk \\
    psyywt@nottingham.ac.uk \\
    psyhz4@nottingham.ac.uk \\
\end{minipage}

\textbf{Description of Team Skills}\\
Our interest in this project stems from the fact that whilst we think this is one of the more challenging projects, it will also be one of the most educational because we will have the opportunity to learn a lot from it. Our team has a fairly extensive amount of experience in programming - especially in personal projects and general programming outside the university.

Both Pedro and Jonny use Linux exclusively for development and have used it for a server in the last couple of hackathons that they attended, most recently in Barcelona where they they finished as finalists at HackUPC - the largest Hackathon in Europe. All of us have experience with databases - most noticeably Holly who has studied databases both at University and at the internship she attended last summer with a company in Hong Kong. Jonathan has also had some experience writing a \href{https://github.com/bonniejools/dumbo}{database explorer for Linux}.

Two of our members, Pedro Santos and Jonathan Balls, have worked together at a programming competition finishing fourth at the most recent programming competition hosted here at Nottingham and are thinking of attending the UKIEPC challenge on 29th October. This style of programming is heavily data processing orientated as almost all puzzles revolve around handling unknown input data as effiently and as quickly as possible.

Several of our members have strong backgrounds in mathematics which we feel will be useful in the context of signal handling and processing as well as 3D programming. Huachen especially has good mathematical knowledge but also Samuel and Jonathan who have A-levels in Further Maths. Their experience with matrices will be important for rendering 3D scenes especially when tilting, panning etc. is required of the camera.

OpenGL is not one of our strong points however Jonathan has experimented with it whilst trying to create an asteroid game in OpenGL and C++. We all feel like this project would be the perfect opportunity to learn more about it and gain more understanding about 3D programming.

\\
\textbf{Highly Desirable and Desirable Skills}\\
\begin{minipage}[t]{0.5\textwidth}
    \textbf{Date of Submission of EoI}\\
    \textbf{Date of Pitch}\\
    \textbf{Notification of award}\\
    
\end{minipage}
\begin{minipage}[t]{0.5\textwidth}
    17 October 2016\\
    21 October 2016\\
    28 October 2016\\
\end{minipage}

\end{document}

